\documentclass[a4paper,10pt]{article}
\usepackage[margin=1in]{geometry}

\usepackage{booktabs}
\usepackage{makecell}
\usepackage{calc}
\usepackage{caption}
\usepackage{lmodern}
\usepackage[utf8]{inputenc}
\usepackage[T1]{fontenc}
\usepackage[hyphens]{url}
\usepackage{parskip}
\usepackage[usenames, dvipsnames]{xcolor}
\usepackage[big]{layaureo} % better formatting of the A4 page, an alternative is fullpage
\usepackage{supertabular}  % for Grades
\usepackage{titlesec}      % custom \section
\usepackage[absolute]{textpos}
\usepackage[norule]{footmisc}
\usepackage{footnote}
\makesavenoteenv{tabular}
\RequirePackage{color, graphicx}

\usepackage{hyperref}
\definecolor{linkcolour}{rgb}{0,0.2,0.6}
\hypersetup{colorlinks, breaklinks, urlcolor = linkcolour, linkcolor = linkcolour}

\titleformat{\section}{\Large\scshape\raggedright}{}{0em}{}[\titlerule]
\titleformat{\subsection}{\bfseries\scshape\raggedright}{}{0em}{}[]
\titlespacing{\section}{0pt}{3pt}{3pt}
\titlespacing*{\subsection}{0pt}{3pt}{0pt}

\setlength{\TPHorizModule}{30mm}
\setlength{\TPVertModule}{\TPHorizModule}
\textblockorigin{2mm}{0.65\paperheight}
\setlength{\parindent}{0pt}

\hyphenation{im-pre-se}

\newcommand{\tabitem}{$\;\circ\;$}
\renewcommand{\labelitemi}{$\circ$}

\newcommand{\cpp}{C\protect\scalebox{0.8}{\protect\raisebox{0.4ex}{++}}}
\renewcommand\#{\protect\scalebox{0.8}{\protect\raisebox{0.4ex}{\char"0023}}}


%--------------------BEGIN DOCUMENT----------------------
\begin{document}

\pagestyle{empty} % non-numbered pages

%--------------------TITLE-------------
\par{
  \centering{
    \Huge \textsc{Gabriel Bastos}
	}
  \bigskip
  \par
}

%--------------------SECTIONS-----------------------------------
\section{About {\normalfont\&} Contact}

\begin{tabular}{rl}
  \textsc{Name:}      & Gabriel Silva Bastos \\
  \textsc{Birthdate:} & 5 Jun 1997 \\
  \textsc{Address:}   & Belo Horizonte -- MG -- Brazil. \\
  \textsc{Phone:}     & \texttt{+} 55 (31) 99286\,--\,2772 \\
  \textsc{e-mail:}    & \href{mailto:gabriel.s.b@live.com}{gabriel.s.b@live.com}
\end{tabular}


\section{Work Experience}
\begin{tabular}{r|p{12.3cm}}
  \textsc{Nov. 2020} & \textsc{MAV Tecnology} \\
  \textsc{Apr. 2019} & Software Engineer \\[5pt]
  & Activities and technologies: \\
  & \tabitem Maintenance of legacy systems. \\
  & \tabitem \cpp, Java and Lua programming. \\
  & \tabitem Microservices architecture, docker. \\

  \multicolumn{2}{c}{} \\
  \textsc{Apr. 2019} & \textsc{Federal University of Minas Gerais} \\
  \textsc{Mar. 2018} & Undergratuate research -- Speed laboratory \\[5pt]
  & Activities: \\
  & \tabitem IoT malware analysis. \\
  & \tabitem Scientific paper writing. \\
  & Topics: \\
  & \tabitem Data mining and analysis, linux infrastructure. \\
  & \tabitem Python and Bash scripting. \\

  \multicolumn{2}{c}{} \\
  \textsc{Mar. 2018} & \textsc{Federal University of Minas Gerais} \\
  \textsc{Mar. 2017} & Computer Programming Prefect \\[5pt]
  & Activities: \\
  & \tabitem Assistance to students in practical activities. \\
  & Topics: \\
  & \tabitem Sequential and combinational circuits. \\
  & \tabitem Scilab programming. \\
  
  \multicolumn{2}{c}{} \\
  \textsc{Sep. 2016} & \textsc{BRAE Biotecnology} \\
  \textsc{Nov. 2015} & Junior systems analyst \\[5pt]
  & Activities: \\
  & \tabitem \makecell[lt] {
              Desktop software development for an
              \href{http://www.ferox.vet.br/pt-br/produtos/ecg-veterinario.aspx}{electrocardiograph}.
             }\\
  & \tabitem \makecell[lt]{
              Desktop software development for a
              \href{http://www.ferox.com.br/pt-br/produtos/monitor-multiparametrico/monitorfx4000.aspx}{multi-parameter monitor}.
             }\\
  & Technologies: \\
  & \tabitem C\# programming. \\
  & \tabitem Microsoft Windows Presentation Foundation. \\
  
  \multicolumn{2}{c}{} \\
  \textsc{Jul. 2015} & \textsc{BRAE Biotecnology} \\
  \textsc{Feb. 2015} & Internship -- Electronics technician \\[5pt]
  & Activities: \\
  & \tabitem Co-development of an \href{http://www.ferox.vet.br/pt-br/produtos/ecg-veterinario.aspx}{electrocardiograph}. \\
  & \tabitem Firmware software development for the electrocardiograph. \\
  & Technologies: \\
  & \tabitem Keil $\mu$ \kern-5pt Vision embbeded development environment. \\
  & \tabitem Embedded C programming.
\end{tabular}


\section{Education}
\begin{tabular}{r|l}
  \textsc{2022} & \textsc{Federal University of Minas Gerais} \\
  \textsc{2016} & Undergraduate in Information Systems \\
  
  \multicolumn{2}{c}{} \\
  \textsc{2015} & \textsc{UFMG's Technical High School} \\
  \textsc{2013} & Electronics technician \\
\end{tabular}


\section{Skills}
\begin{tabular}{r|l}
  \textsc{English} & Advanced reading, writing, hearing and speaking. \\
  
  \multicolumn{2}{c}{} \\
  \textsc{Programming} & I'm familiar with the following programming languages: \\
  \textsc{Languages} & C, \cpp, C\#, Rust, Haskell, F\#, Python, Lua. \\
  
  \multicolumn{2}{c}{} \\
  \textsc{Technologies} & I'm familiar with the following technologies: \\
  & Linux, bash, git, github, docker, emacs.
\end{tabular}


\section{Personal Interests}
I love computers and have been programming since I was 14 years old. \\
Some of the topics of my interest:
\vspace{-3pt}
\begin{itemize}
  \setlength\itemsep{-3pt}
  \item Programming languages and compilers.
  \item Type systems, programming paradigms.
  \item Systems programming.
  \item Clean code and software engineering techniques in general.
  \item Open source software.
\end{itemize} \vspace{5pt}

\subsection{Projects}
Besides personal projects, I also have made contributions to at least 15 open source
repositories. All of my activity can be found at my \href{https://github.com/gahag/}{github}.
Here are some of my projects: \\
\begin{tabular}{r|l}
  \multicolumn{2}{c}{} \\
  \textsc{Hush} & \url{https://github.com/gahag/hush} \\[3pt]
  & Hush is a modern Unix shell based on the Lua programming language. \\
  & Currently a work in progress. \\

  \multicolumn{2}{c}{} \\
  \textsc{DCI} & \url{https://github.com/gahag/dci} \\[3pt]
  & DCI-Closed \footnote{\url{http://hpc.isti.cnr.it/~claudio/papers/2004_FIMI_dci_closed.pdf}}:
  a frequent closed itemset mining algorithm, implemented in Rust.  \\

  \multicolumn{2}{c}{} \\
  \textsc{Bgrep} & \url{https://github.com/gahag/bgrep} \\[3pt]
  & Bgrep is a \textit{grep} spin for binary patterns and files. \\

  \multicolumn{2}{c}{} \\
  \textsc{FSQL} & \url{https://github.com/gahag/fsql} \\[3pt]
  & FSQL is a tool written in haskell to query the file system with SQL-like syntax. \\
  
  \multicolumn{2}{c}{} \\
  \textsc{TPC} & \url{https://github.com/gahag/tpc} \\[3pt]
  & TPC is a template based parser combinator for \cpp. \\
\end{tabular}


\end{document}

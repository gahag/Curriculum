\documentclass[a4paper,10pt]{article}
\usepackage[margin=1in]{geometry}

\usepackage{booktabs}
\usepackage{makecell}
\usepackage{calc}
\usepackage{caption}
\usepackage{lmodern}
\usepackage[utf8]{inputenc}
\usepackage[T1]{fontenc}
\usepackage[hyphens]{url}
\usepackage{parskip}
\usepackage[usenames, dvipsnames]{xcolor}
\usepackage[big]{layaureo} % better formatting of the A4 page, an alternative is fullpage
\usepackage{supertabular}  % for Grades
\usepackage{titlesec}      % custom \section
\usepackage[absolute]{textpos}
\RequirePackage{color, graphicx}

\usepackage{hyperref}
\definecolor{linkcolour}{rgb}{0,0.2,0.6}
\hypersetup{colorlinks, breaklinks, urlcolor = linkcolour, linkcolor = linkcolour}

\titleformat{\section}{\Large\scshape\raggedright}{}{0em}{}[\titlerule]
\titleformat{\subsection}{\bfseries\scshape\raggedright}{}{0em}{}[]
\titlespacing{\section}{0pt}{3pt}{3pt}
\titlespacing*{\subsection}{0pt}{3pt}{0pt}

\setlength{\TPHorizModule}{30mm}
\setlength{\TPVertModule}{\TPHorizModule}
\textblockorigin{2mm}{0.65\paperheight}
\setlength{\parindent}{0pt}

\hyphenation{im-pre-se}

\newcommand{\tabitem}{$\;\circ\;$}
\renewcommand{\labelitemi}{$\circ$}

\newcommand{\cpp}{C\protect\scalebox{0.8}{\protect\raisebox{0.4ex}{++}}}
\renewcommand\#{\protect\scalebox{0.8}{\protect\raisebox{0.4ex}{\char"0023}}}


%--------------------BEGIN DOCUMENT----------------------
\begin{document}

\pagestyle{empty} % non-numbered pages

%--------------------TITLE-------------
\par{
  \centering{
    \Huge \textsc{Gabriel Bastos}
	}
  \bigskip
  \par
}

%--------------------SECTIONS-----------------------------------
\section{Dados Pessoais}

\begin{tabular}{rl}
  \textsc{Nome Completo:} & Gabriel Silva Bastos \\
  \textsc{Idade:}         & 22 anos \\
  \textsc{Endereço:}      & Rua dos Assistentes Sociais, 294 -- Alípio de Melo -- Belo Horizonte. \\
  \textsc{Telefone:}      & (31) 99286\,-\,2772 \\
  \textsc{e-mail:}        & \href{mailto:gabriel.s.b@live.com}{gabriel.s.b@live.com}
\end{tabular}


\section{Experiência Profissional}
\begin{tabular}{r|p{12.3cm}}
  \textsc{Presente} & \textsc{MAV Tecnologia} \\
  \textsc{Abr. 2019} & Analista de Desenvolvimento de Sistemas \\[5pt]
  & Atividades desenvolvidas e tecnologias utilizadas: \\
  & \tabitem Manutenção e correção de bugs em sistemas legados. \\
  & \tabitem Programação \cpp, Java, Lua. \\
  & \tabitem Arquitetura de microserviços, docker. \\

  \multicolumn{2}{c}{} \\
  \textsc{Abr. 2019} & \textsc{Universidade Federal de Minas Gerais} \\
  \textsc{Mar. 2018} & Iniciação científica no laboratório Speed \\[5pt]
  & Atividades desenvolvidas: \\
  & \tabitem Análise de malwares IoT. \\
  & \tabitem Escrita de artigos científicos. \\
  & Conteúdos abordados: \\
  & \tabitem Infraestruturas baseadas em Linux para coleta e análise de dados. \\
  & \tabitem Programação Python e Bash. \\

  \multicolumn{2}{c}{} \\
  \textsc{Mar. 2018} & \textsc{Universidade Federal de Minas Gerais} \\
  \textsc{Mar. 2017} & Monitor da disciplina Programação de Computadores \\[5pt]
  & Atividades desenvolvidas: \\
  & \tabitem Assistência aos alunos nas atividades práticas da disciplina. \\
  & Conteúdos abordados: \\
  %% & \tabitem Circuitos sequenciais e combinacionais. \\
  & \tabitem Programação em Scilab. \\
  
  \multicolumn{2}{c}{} \\
  \textsc{Set. 2016} & \textsc{BRAE Biotecnologia} \\
  \textsc{Nov. 2015} & Analista de sistemas júnior \\[5pt]
  & Atividades desenvolvidas: \\
  & \tabitem \makecell[lt] {
              Manutenção e desenvolvimento do \textit{software desktop} do
              \href{http://www.ferox.vet.br/pt-br/produtos/ecg-veterinario.aspx}{eletrocardiógrafo}.
             }\\
  & \tabitem \makecell[lt]{
              Manutenção e desenvolvimento dos \textit{softwares desktop} e \textit{mobile} do \\
              \href{http://www.ferox.com.br/pt-br/produtos/monitor-multiparametrico/monitorfx4000.aspx}{monitor multi-paramétrico}.
             }\\[-3pt]
  & Tecnologias utilizadas: \\
  & \tabitem Programação C\#, \textit{Microsoft Visual Studio}. \\
  & \tabitem \textit{Microsoft Windows Presentation Foundation}. \\
  & \tabitem \textit{Xamarin}. \\
  
  \multicolumn{2}{c}{} \\
  \textsc{Jul. 2015} & \textsc{BRAE Biotecnologia} \\
  \textsc{Fev. 2015} & Estágio -- Técnico em eletrônica \\[5pt]
  & Atividades desenvolvidas: \\
  & \tabitem Participação no desenvolvimento do projeto eletrônico de um \href{http://www.ferox.vet.br/pt-br/produtos/ecg-veterinario.aspx}{eletrocardiógrafo}. \\
  & \tabitem \makecell[lt]{
              Projeto e desenvolvimento do firmware responsável pela comunicação do \\
              eletrocardiógrafo com o computador.
             } \\
  & Tecnologias utilizadas: \\
  & \tabitem Ambiente de desenvolvimento para plataformas embarcadas \textit{Keil $\mu$ \kern-5pt Vision}. \\
  & \tabitem Linguagem C.
\end{tabular}


\section{Formação Acadêmica}
\begin{tabular}{r|l}
  \textsc{2021} & \textsc{Universidade Federal de Minas Gerais} \\
  \textsc{2016} & Cursando bacharelado em Sistemas de Informação \\
  
  \multicolumn{2}{c}{} \\
  \textsc{2015} & \textsc{Colégio Técnico da UFMG} \\
  \textsc{2013} & Técnico em eletrônica \\
\end{tabular}


\section{Competências}
\begin{tabular}{r|l}
  \textsc{Inglês} & Leitura, escrita e audição fluentes. \\
  & Pronúncia intermediária. \\
  
  \multicolumn{2}{c}{} \\
  \textsc{Linguagens de} & Possuo familiaridade com as seguintes linguagens: \\
  \textsc{Programação}& C, \cpp, C\#, Rust, Haskell, F\#, Python, Lua. \\
  
  \multicolumn{2}{c}{} \\
  \textsc{Tecnologias} & Possuo familiaridade com as seguintes tecnologias: \\
  & git, github, docker, bash, linux, systemd, visual studio, .NET framework, \\
  & emacs, parsec(Haskell).
\end{tabular}


\section{Interesses Pessoais}
Sou apaixonado por tecnologia, e programo desde os 14 anos. \\
Algumas das áreas e tecnologias que despertam meu interesse:
\vspace{-3pt}
\begin{itemize}
  \setlength\itemsep{-3pt}
  \item Linguagens de programação e compiladores, sistemas de tipos, paradigmas de programação.
  \item Análise estática e otimização de programas, \emph{LLVM}, \emph{JIT compilers}.
  \item Ferramentas e técnicas de programação e engenharia de \emph{software} em geral.
\end{itemize}

\subsection{Projetos Pessoais}
Todos meus projetos pessoais podem ser encontrados na minha \href{https://github.com/gahag/}{página no github}. \\[10pt]
\begin{tabular}{r|l}
  \textsc{FSQL} & \url{https://github.com/gahag/fsql} \\[3pt]
  & FSQL é uma ferramenta escrita em Haskell para realizar consultas \\
  & ao sistema de arquivos utilizando uma sintaxe semelhante a SQL. \\
  
  \multicolumn{2}{c}{} \\
  \textsc{TPC} & \url{https://github.com/gahag/tpc} \\[3pt]
  & TPC é um parser combinator para \cpp\ baseado em templates. \\

  \multicolumn{2}{c}{} \\
  \textsc{Bgrep} & \url{https://github.com/gahag/bgrep} \\[3pt]
  & Bgrep é um variante do clássico \textit{grep} para padrões e arquivos binários. \\
\end{tabular}


\end{document}

\documentclass[a4paper,10pt]{article}
\usepackage[margin=1in]{geometry}

\usepackage{booktabs}
\usepackage{makecell}
\usepackage{calc}
\usepackage{caption}
\usepackage{lmodern}
\usepackage[utf8]{inputenc}
\usepackage[T1]{fontenc}
\usepackage[hyphens]{url}
\usepackage{parskip}
\usepackage[usenames, dvipsnames]{xcolor}
\usepackage[big]{layaureo} % better formatting of the A4 page, an alternative is fullpage
\usepackage{supertabular}  % for Grades
\usepackage{titlesec}      % custom \section
\usepackage[absolute]{textpos}
\RequirePackage{color, graphicx}

\usepackage{hyperref}
\definecolor{linkcolour}{rgb}{0,0.2,0.6}
\hypersetup{colorlinks, breaklinks, urlcolor = linkcolour, linkcolor = linkcolour}

\titleformat{\section}{\Large\scshape\raggedright}{}{0em}{}[\titlerule]
\titleformat{\subsection}{\bfseries\scshape\raggedright}{}{0em}{}[]
\titlespacing{\section}{0pt}{3pt}{3pt}
\titlespacing*{\subsection}{0pt}{3pt}{0pt}

\setlength{\TPHorizModule}{30mm}
\setlength{\TPVertModule}{\TPHorizModule}
\textblockorigin{2mm}{0.65\paperheight}
\setlength{\parindent}{0pt}

\hyphenation{im-pre-se}

\newcommand{\tabitem}{$\;\circ\;$}
\renewcommand{\labelitemi}{$\circ$}

\newcommand{\cpp}{C\protect\scalebox{0.8}{\protect\raisebox{0.4ex}{++}}}
\renewcommand\#{\protect\scalebox{0.8}{\protect\raisebox{0.4ex}{\char"0023}}}


%--------------------BEGIN DOCUMENT----------------------
\begin{document}

\pagestyle{empty} % non-numbered pages

%--------------------TITLE-------------
\par{
  \centering{
    \Huge \textsc{Gabriel Bastos -- Curriculum}
	}
  \bigskip
  \par
}

%--------------------SECTIONS-----------------------------------
\section{About {\normalfont\&} Contact}

\begin{tabular}{rl}
  \textsc{Name:}     & Gabriel Silva Bastos \\
  \textsc{Birthday:} & 5 Jun 1997 \\
  \textsc{Address:}  & Rua dos Assistentes Sociais, 294 -- Alípio de Melo -- Belo Horizonte. \\
  \textsc{Phone:}    & (31) 99286\,-\,2772 \\
  \textsc{e-mail:}   & \href{mailto:gabriel.s.b@live.com}{gabriel.s.b@live.com}
\end{tabular}


\section{Work Experience}
\begin{tabular}{r|p{12.3cm}}
  \textsc{Dec. 2017} & \textsc{Universidade Federal de Minas Gerais} \\
  \textsc{Mar. 2017} & Monitor da disciplina Programação de Computadores \\[5pt]
  & Atividades desenvolvidas: \\
  & \tabitem Assistência aos alunos nas atividades práticas da disciplina. \\
  & Conteúdos abordados: \\
  & \tabitem Circuitos sequenciais e combinacionais. \\
  & \tabitem Programação em Scilab. \\
  
  \multicolumn{2}{c}{} \\
  \textsc{Set. 2016} & \textsc{BRAE Biotecnologia} \\
  \textsc{Nov. 2015} & Junior systems analyst \\[5pt]
  & Atividades desenvolvidas: \\
  & \tabitem \makecell[lt] {
              Manutenção e desenvolvimento do software desktop de operação do \\
              \href{http://www.ferox.vet.br/pt-br/produtos/ecg-veterinario.aspx}{eletrocardiógrafo}.
             }\\
  & \tabitem \makecell[lt]{
              Manutenção e desenvolvimento dos softwares desktop e mobile de operação do \\
              \href{http://www.ferox.com.br/pt-br/produtos/monitor-multiparametrico/monitorfx4000.aspx}{monitor multi-paramétrico}.
             }\\[-3pt]
  & Tecnologias utilizadas: \\
  & \tabitem Ambiente de desenvolvimento Microsoft Visual Studio. \\
  & \tabitem Linguagem C\#. \\
  & \tabitem Microsoft Windows Presentation Foundation. \\
  & \tabitem Xamarin. \\
  
  \multicolumn{2}{c}{} \\
  \textsc{Jul. 2015} & \textsc{BRAE Biotecnologia} \\
  \textsc{Fev. 2015} & Internship -- Electronics technician \\[5pt]
  & Atividades desenvolvidas: \\
  & \tabitem Participação no desenvolvimento do projeto eletrônico de um \href{http://www.ferox.vet.br/pt-br/produtos/ecg-veterinario.aspx}{eletrocardiógrafo}. \\
  & \tabitem \makecell[lt]{
              Projeto e desenvolvimento do firmware responsável pela comunicação do \\
              eletrocardiógrafo com o computador.
             } \\
  & Tecnologias utilizadas: \\
  & \tabitem Ambiente de desenvolvimento para plataformas embarcadas Keil $\mu$ \kern-5pt Vision. \\
  & \tabitem Linguagem C.
\end{tabular}


\section{Education}
\begin{tabular}{r|l}
  \textsc{2020} & \textsc{Universidade Federal de Minas Gerais} \\
  \textsc{2016} & Bachelor in Information Systems \\
  
  \multicolumn{2}{c}{} \\
  \textsc{2015} & \textsc{Colégio Técnico da UFMG} \\
  \textsc{2013} & Electronics technician \\
\end{tabular}


\section{Skills}
\begin{tabular}{r|l}
  \textsc{English} & Proficient reading, writing and hearing. \\
  & Intermediary pronounce. \\
  
  \multicolumn{2}{c}{} \\
  \textsc{Programming} & I'm familiar with the following languages: \\
  \textsc{Languages} & C, \cpp, C\#, Haskell, F\#, Python. \\
  
  \multicolumn{2}{c}{} \\
  \textsc{Technologies} & I'm familiar with the following technologies: \\
  & git, github, bash, linux, visual studio, .NET framework, emacs, \\
  & stack(Haskell), parsec(Haskell).
\end{tabular}


\section{Personal Interests}
I love technology, and have been programming since I was 14 years old. \\
Some of the topics and technologies of my interest:
\vspace{-3pt}
\begin{itemize}
  \setlength\itemsep{-3pt}
  \item Programming languages and compilers.
  \item Parsers, type systems, programming paradigms.
  \item Static analysis and optimizations, LLVM, JIT compilers.
  \item Programming tools and software engineering techniques in general.
\end{itemize}

\subsection{Projects}
All of my projects can be found at my \href{https://github.com/gahag/}{github}. \\[10pt]
\begin{tabular}{r|l}
  \textsc{FSQL} & \url{https://github.com/gahag/fsql} \\[3pt]
  & FSQL is a tool written in haskell to query the file system with SQL-like syntax. \\
  
  \multicolumn{2}{c}{} \\
  \textsc{TPC} & \url{https://github.com/gahag/tpc} \\[3pt]
  & TPC is a template based parser combinator for \cpp. \\
\end{tabular}


\end{document}
